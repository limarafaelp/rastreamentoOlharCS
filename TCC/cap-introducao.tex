\chapter{Introdução}

Um dos problemas enfrentados por pessoas com deficiência motora é a falta de acesso à informação devido à impossibilidade ou dificuldade para usar dispositivos eletrônicos. Uma possível solução para esse problema seria usar o movimento dos olhos para interagir com o mundo exterior. O uso de técnicas para identificar onde determinada pessoa está olhando é chamado de rastreamento de olhar.

Para rastrear o olhar geralmente uma câmera é posicionada na frente de um dos olhos. A partir da imagem obtida pela câmera, técnicas de processamento de imagens são usadas para identificar a posição da íris e, a partir disso, identificar a direção em que o usuário está olhando. No entanto, o rastreamento de olhar é dificultado por vários fatores, como mudanças na iluminação do lugar e oclusão parcial dos olhos causada por alguma expressão facial. Portanto, não é uma tarefa trivial mas, apesar dessas dificuldades, deve ser executada em tempo real.

Uma possível solução seria processar não a imagem original, mas uma imagem menor e com as mesmas características desejadas encontradas na primeira.

Uma imagem pode ser representada como uma combinação linear de vetores que representam senoides de frequências diferentes. Empiricamente é assumido que frequências mais altas influenciam pouco no resultado final e, por esse motivo, em muitas aplicações essas frequências são descartadas para compressão de imagens, ou seja, reduzir o 'espaço físico' usado para armazenar o arquivo.

A abordagem sugerida pelo Compressive Sensing (CS) é um pouco diferente: em vez de assumir quais vetores podem ser descartados, é assumido que uma determinada quantidade de vetores exerce pouca ou nenhuma influência no sinal. Chamamos de esparso um sinal que depende apenas de uma pequena quantidade de vetores da base. Quando o sinal depende de vários vetores, porém apenas uma pequena quantidade influencia significativamente no resultado final, esse sinal é conhecido como `compressive'. Em muitos casos CS permite reconstruir, sem perda de qualidade, uma imagem usando menos amostras do que as necessárias para reconstruir imagens usando técnicas tradicionais de compressão de imagens.