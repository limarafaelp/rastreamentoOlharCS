\chapter{Rastreamento de olhar}

%Rastreamento de olhar consiste em técnicas para determinar a direção do olhar, ou seja, a partir da posição da íris identificar para onde uma pessoa está olhando.

Rastreamento de olhar consiste em determinar a direção do olhar, ou seja, a partir da posição do olho identificar para onde uma pessoa está olhando em determinado instante e também mudanças na direção do olhar em determinado período \cite{lupung}.

Abaixo listamos algumas aplicações de rastreamento de olhar.

\section{Aplicações}

Podemos usar o rastreamento de olhar para construir interfaces para ajudar pessoas com dificuldades motoras a interagir com computadores \cite{lidade}. \newc{Kurauchi et al.\cite{kurauchi2016eyeswipe} desenvolveram} um teclado virtual que permite a digitação através do olhar, como mostra a Figura $\ref{fig:eyeswipe}$.

\begin{figure}
\centering
\includegraphics[scale=.3]{imagens/teclado.png}
\caption{Um teclado virtual controlado pelo olhar permite a interação de pessoas com dificuldades motoras com o computador. Reproduzido de \cite{kurauchi2016eyeswipe}.}
\label{fig:eyeswipe}
\end{figure}

Através da análise da posição do olhar é possível avaliar a influência de um anúncio sobre a atenção dos consumidores e, assim, ajudar na criação de propagandas mais eficientes \cite{duchowski2002breadth}. A Figura \ref{fig:propaganda} mostra as regiões onde as pessoas mais prestam atenção ao observar um anúncio.

\afterpage{
\begin{figure}
\begin{center}
\includegraphics[scale=.4]{imagens/dolcegabbana_heatmap.jpg}
\caption[]{O rastreamento de olhar pode ser usado para a avaliar a eficiência de uma propaganda. Regiões coloridas indicam onde as pessoas prestam mais atenção no anúncio. Reproduzido de \protect \small{\url{http://www.businessinsider.com.au/eye-tracking-heatmaps-2014-7}}}
\label{fig:propaganda}
\end{center}
\end{figure}
%\footnotetext{\url{http://www.businessinsider.com.au/eye-tracking-heatmaps-2014-7}}
}

O olhar também pode ser usado como um indicador de usabilidade de interfaces de \textit{software}, sendo usado em estudos de interação humano-computador (IHC) \cite{duchowski2002breadth}. Um estudo feito por \cite{nielsen2007fancy} mostra que usuários tendem \newc{a} ignorar informações que parecem propagandas em um site. No caso, solicitaram para os participantes encontrarem a população dos Estados Unidos e $86\%$ dos participantes falharam, mesmo com o número localizado no topo da página e escrito em vermelho. A Figura $\ref{fig:censo}$ mostra as regiões mais observadas pelos usuários.

%\begin{figure}
%\begin{center}
%\includegraphics[scale=.7]{imagens/census-homepage-heatmap.jpg}
%\caption{Regiões vermelhas são as mais observadas no site por participantes procurando a população dos Estados Unidos. Note que a região com essa informação, no canto superior direito, é parcipalmente observada, indicando que foi ignorada pelos usuários. Reproduzido de \cite{nielsen2007fancy}.}
%\label{fig:censo}
%\end{center}
%\end{figure}

\afterpage{
\begin{figure}
\centering
	\begin{subfigure}[b]{1\textwidth}
		\centering
		\includegraphics[scale=.7]{imagens/census-homepage-small.jpg}
		\caption{}
	\end{subfigure}\\
	\begin{subfigure}[b]{1\textwidth}
		\centering
		\includegraphics[scale=.7]{imagens/census-homepage-heatmap.jpg}
		\caption{}
	\end{subfigure}
	\caption{\newc{{\bf a} Imagem original do site. Regiões vermelhas em {\bf (b)}} são as mais observadas no site por participantes procurando a população dos Estados Unidos. Note que a região com essa informação, no canto superior direito, é parcipalmente observada, indicando que foi ignorada pelos usuários. Reproduzido de \cite{nielsen2007fancy}.}
	\label{fig:censo}
\end{figure}
%\footnotetext{\url{http://www.businessinsider.com.au/eye-tracking-heatmaps-2014-7}}
}
\section{Técnicas de rastreamento de olhar}

Técnicas de rastreamento de olhar determinam a posição do olhar a partir do movimento dos olhos. Algumas partes do olho humano, que estão representadas na Figura $\ref{fig:olho}$,  são as seguintes \cite{lidade}:

\afterpage{
%assim o footnote funciona http://blog.peschla.net/2012/11/latex-footnotes-in-captions/
\begin{figure}
\begin{center}
\includegraphics[scale=1]{imagens/eye.jpg}
\caption[]{Regiões do olho. Adaptado de \protect \small{\url{http://commons.wikimedia.org/wiki/File:My_eye.jpg}}}
\label{fig:olho}
\end{center}
\end{figure}
%\footnotetext{\url{http://commons.wikimedia.org/wiki/File:My_eye.jpg}}
}

\begin{itemize}
\item \textbf{pupila}: a abertura que permite a entrada de luz no olho.
\item \textbf{íris}: o músculo colorido que controla o diâmetro da pupila.
\item \textbf{esclera}: tecido branco protetor que envolve o restante do olho.
\item \textbf{limbo}: o contorno entre a íris e a esclera.
\end{itemize}

%[sic: no texto estava Eye location/tracking techniques can be...]
%Técnicas de rastreamento de olhar podem ser divididas em três modalidades \cite{valenti2009webcam}.

~ %para evitar um erro estranho de missing item

Técnicas de rastreamento de olhar podem ser divididas em três modalidades \cite{valenti2009webcam}:

\begin{enumerate}
\item {\bf Eletro-oculografia}, que registra diferenças de potencial elétrico na pele ao redor da cavidade ocular.
\item {\bf Lente de contato}, uma bobina é instalada no olho sobre uma lente de contato e a posição do olho é estimada ao medir o campo eletromagnético. \cite{duchowski2007eye}. Este método é o mais preciso para medir movimentos do olho, porém é o mais \oldc{invasivo}\newc{desconfortável} \cite{duchowski2007eye}.

%sic: no texto estava oculografia baseada em foto ou vídeo
\item  {\bf Rastreamento baseado em vídeo}, que usa técnicas de processamento de imagens para identificar a posição da pupila nas imagens do olho. \newc{Esse método é mais confortável que os demais métodos e razoavelmente preciso.}
\end{enumerate}

\oldc{O rastreamento baseado em vídeo é a modalidade de rastreamento de olhar menos invasiva \cite{valenti2009webcam}.}

Dois tipos de imagem são comumente usados no rastreamento baseado em vídeo: imagens no espectro visual ou imagens em infravermelho.

No rastreamento \oldc{pelo espectro visual}\newc{baseado em vídeo}, a luz refletida pelos olhos é registrada. Neste caso, a eficiência do rastreamento depende das condições de iluminação do ambiente, o que torna o processo complicado $\cite{lidade}$.

No caso de imagens em infravermelho, não temos esse problema, pois o olho é constantemente iluminado por uma fonte de luz infravermelha, que não é percebida pelo usuário. Uma vantagem desta abordagem é que a pupila reflete parte da luz infravermelha recebida, sendo a região mais brilhante do olho na imagem $\cite{lidade}$. Devido ao seu tamanho e posição, a pupila tem menor chance de ser parcialmente ocultada pelos  cílios do que a esclera, outra região que reflete o infravermelho. %pupila será sempre a região mais brilhante na imagem $\cite{lupung}$: se a fonte de luz estiver alinhada com o olho, a pupila aparecerá branca, caso contrário, estará preta na imagem.

A principal desvantagem é que não é possível usar esse tipo de rastreador ao ar livre durante o dia devido à luz infravermelha do ambiente $\cite{lidade}$.

As técnicas de rastreamento de olhar também variam de acordo com a localização da câmera, que pode ser instalada junto à cabeça do usuário (\textit{head-mounted}) ou remotamente. No caso do rastreamento \textit{head-mounted}, o usuário deve usar um acessório equipado com uma câmera que registra as imagens do olho. A figura $\ref{fig:pupil}$ mostra um rastreador desse tipo

\afterpage{
%assim o footnote funciona http://blog.peschla.net/2012/11/latex-footnotes-in-captions/
\begin{figure}
\begin{center}
\includegraphics[scale=.7]{imagens/pupil.png}
\caption[]{Um rastreador de olhar acoplado à cabeça. Reproduzido de \protect \small{\url{https://pupil-labs.com/blog/2014-01/new-pupil-pro-headset-capture-software-0-3-7/}} }
\label{fig:pupil}
\end{center}
\end{figure}
%\footnotetext{\url{https://pupil-labs.com/blog/2014-01/new-pupil-pro-headset-capture-software-0-3-7/}}
}

Uma vantagem do rastreamento \textit{head-mounted} é que a câmera se move com a cabeça, então mudanças de pose não causam mudanças da posição da pupila na imagem. A desvantagem é a necessidade de usar um equipamento acoplado à cabeça. Por estar muito próxima do olho, a câmera do olho pode ocultar parte do campo de visão, dificultando a interação com o usuário.

\section{Características \newc{relevantes}}

Em algumas aplicações de rastreamento de olhar não estamos interessados apenas na direção do olhar em determinado instante, mas também podemos querer observar outras características relacionadas ao movimento do olhos. As características mais \oldc{específicas}\newc{relevantes} são \cite{lupung}:

\begin{itemize}
\item {\bf Fixação e sacada:} Em \newc{1879}, Emile Java (oftalmologista francês) observou que os movimentos do olho não ocorrem de forma contínua, e sim de movimentos rápidos, chamados de sacadas, seguidos por breves pausas, conhecidas como fixações \cite{lupung};

\item {\bf Área de interesse:} Região no ambiente que está presente no campo visual e que é de interesse em determinada pesquisa;

\item {\bf Duração do olhar:} Duração de um intervalo de tempo em que uma série de fixações está dentro de uma área de interesse;

\item {\bf Caminho de varredura:} localização espacial de uma sequência de fixações.
\end{itemize}