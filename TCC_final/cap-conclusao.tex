\chapter{Conclusão}

Neste trabalho estudamos conceitos de rastreamento de olhar, a teoria de \textit{Compressive Sensing} (CS) e desenvolvemos um programa de rastreamento de olhar.

%Observamos que CS é útil tanto para recuperar sinais esparsos, como também para comparar imagens.

CS é útil para recuperar sinais esparsos, resolvendo com alta probabilidade um problema NP. Uma variação do CS pode ser usada para comparar imagens, usamos esta variação para construir um rastreador de olhar.

Elaboramos um experimento para testar a precisão e acurácia do rastreador e notamos que o programa apresenta algumas limitações, por exemplo, o programa não consegue estimar o olhar para determinada região se o participante piscar durante a coleta da amostra correspondente. Apesar das limitações, o experimento apresentou um resultado semelhante ao de um rastreador comercial.

Este trabalho contribuiu para a formação do aluno em Matemática Aplicada pois, durante o ano o aluno estudou a técnica de \textit{Compressive Sensing}, que apresenta resultados não triviais, e a aplicou em um problema de processamento de imagens.
